\documentclass[
  12pt, % Fontsize
  a4paper, % papersize
  oneside, % For twosided documents
  openany, 
  numbers=noenddot, % No final dots in Sectionnumbers, e.g 1.2 instead of 1.2.
  BCOR=5mm, % Correction length for lost space from binding
  parskip=half*, %No indent but spacing between paragraphs
  thesis, % type of document
]{bfhbook}


% Test Template for bfhbook.cls
\usepackage[T1]{fontenc}
% Coding 
\usepackage[utf8]{inputenc}
% Language setting
\usepackage[german]{babel}
\usepackage[export]{adjustbox}

% \usepackage{fonttable}
% Hyperref
\usepackage[                
  pdftex,                  % for PDF
  colorlinks=true,         % colored links
  linkcolor=black,         % color for links
  citecolor=black,         % color for references
  urlcolor=black,          % color for url 
  bookmarks=true
]{hyperref}              

\usepackage{booktabs} % For nicer tables
\usepackage{threeparttable} % Table-Captions having the same width than the table
\usepackage[singlelinecheck=off]{caption}
\usepackage{siunitx} % Scientific Units and number setting
\usepackage{listings} % For Program-Code
\usepackage{enumitem}
\setlist[description]{style=nextline}

\usepackage{minted}
\setminted[arduino]{
    linenos=true,
    breaklines=true,
    encoding=utf8,
    fontsize=\footnotesize,
    xleftmargin=7mm
}
\setminted[shell]{
	frame=lines,
	framesep=5mm
}
\usemintedstyle{manni}

\usepackage{caption}
\captionsetup[figure]{font=footnotesize, labelfont=small}
\newcommand{\source}[1]{\caption*{Quelle: {#1}} }

\usepackage{xcolor}

\usepackage[export]{adjustbox}
\usepackage[document]{ragged2e} % left-alignment for text

\usepackage{glossaries}

% definition for directory tree with forest
\usepackage[edges]{forest}

\definecolor{foldercolor}{RGB}{124,166,198}

\tikzset{pics/folder/.style={code={%
    \node[inner sep=0pt, minimum size=#1](-foldericon){};
    \node[folder style, inner sep=0pt, minimum width=0.3*#1, minimum height=0.6*#1, above right, xshift=0.05*#1] at (-foldericon.west){};
    \node[folder style, inner sep=0pt, minimum size=#1] at (-foldericon.center){};}
    },
    pics/folder/.default={20pt},
    folder style/.style={draw=foldercolor!80!black,top color=foldercolor!40,bottom color=foldercolor}
}

\forestset{is file/.style={edge path'/.expanded={%
        ([xshift=\forestregister{folder indent}]!u.parent anchor) |- (.child anchor)},
        inner sep=1pt},
    this folder size/.style={edge path'/.expanded={%
        ([xshift=\forestregister{folder indent}]!u.parent anchor) |- (.child anchor) pic[solid]{folder=#1}}, inner xsep=0.6*#1},
    folder tree indent/.style={before computing xy={l=#1}},
    folder icons/.style={folder, this folder size=#1, folder tree indent=3*#1},
    folder icons/.default={12pt},
}

% end definition directory tree

%%%%%%%%%%%%%%%%%%%%%%%%%%%%%%%%%%%
% Settings 
%%%%%%%%%%%%%%%%%%%%%%%%%%%%%%%%
% Type?? (Lecture Notes, BSc Thesis, Master Thesis, . . .) 
% Use Variables \BSc, \Master, etc. for language support
\type{Master Thesis}
% Author(s)
\author{Marc Habegger}
% Title
\title{Explainable AI}
% Short Title, will be used in the footline
\shorttitle{MAS Data Science Master Thesis}
% Subtitle
\subtitle{Stand der Forschung und Technik}
% Titlepicture
% \titlepicture{Bilder/Titel.png}
%%

% Topic of Study
\degreeprogramme{MAS Data Science}
% Expert
\expert{Max Kleiner}
% Version
\version{1.0}
% Date
\date{\today} % Or any other possible date

% Departement
% Use Variable for language support
%\TI

% Semester
% Use Variable for language support
%\semester

% Logo(s)

% Colors
% Secondary Color for Graphics, Tables etc.
% Naming: BFH*Color*light|middle|dark, e.g. BFHGreendark, BFHBluelight, etc.
% Possible Color Values: Green, Blue, Purple, Brown 
\newcommand{\seccolor}{BFHLightGreen} 
\newcommand{\imgText}[2]{
\begin{center}
    \begin{minipage}[t]{0.6\textwidth}
\includegraphics[width=10cm, left, valign=t]{Bilder/#1}
	\end{minipage}\hfill
    \begin{minipage}[t]{0.4\textwidth}
  #2
    \end{minipage}
\end{center}
}

\setcounter{secnumdepth}{4}
\setcounter{tocdepth}{4}

% Variablen für diese Arbeit
\newcommand{\compImgSize}{4cm}

% Glossar Einträge
\makeindex
\makeglossaries

\newglossaryentry{XAI}
{
	name=Explainable artificial intelligence,
	description={deutsch erklärbare künstliche Inteligenz, Methodiken um Menschen die Vorhersagen durch Modelle des maschinellen Lernens zu erläutern.  \break
	\url{https://en.wikipedia.org/wiki/Explainable_artificial_intelligence}}
}

\newglossaryentry{DNN}
{
    name=Deep Neural Network,
    description={deutsch tiefes lernen, Bezeichnet Neuronale Netze mit vielen Zwischenschichten.\break 
    \url{https://en.wikipedia.org/wiki/Deep_learning\#Deep_neural_networks}}
}

\newglossaryentry{LRP}                                 
{
	name=Layer-wise Relevance Propagation,
	description={test}                                   
}                             

\begin{document}
                         
\maketitle
%**************************************************************************
%\frontmatter % preliminary parts

\tableofcontents
\sloppy
%%%%%%%%%%%%%%%%%%%%%%%%
% Introduction
%**************************************************************************
\mainmatter % The main part
%**************************************************************************
%\part{Part One}

\chapter{Einleitung}
\Gls{XAI}
\chapter{Eine Frage der Perspektive}
\section{Erklärbarkeit}
\subsection{Unterschiedliche Ziele}
Je nach Komponente einer ML Lösung ergeben sich unterschiedliche Anforderungen an die Erklärbarkeit: \cite{XAI2018}

\begin{description}
\item[Daten]
Aus der Sicht der Daten interessiert vor allem welcher Teil der Daten für das Ergebnis die Grösste Relevanz hat

\item[Modell]
Kann man aus dem Modell Muster für eine bestimmte Kategorie ableiten?

\item[Vorhersage]
Erklärung weshalb ein bestimmtes Muster in den Daten zu der beobachteten Klassifizierung geführt hat
\end{description}

\section{Transparenz}
\section{Betroffene Parteien}
Je nach Interessengruppe bestehen unterschiedliche Anforderungen an die Erklärbarkeit einer ML Anwendung. Nach \cite{Ras2018} werden dabei folgende Gruppierungen unterschieden:
\begin{itemize}
  \item Experten
  	\begin{itemize}
  		\item Forscher Entwickelt neue Methoden und Algorithmen für das ML, verbessert bestehende Algorithmen
  		\item Entwickler Setzt bestehende Methodiken und Algorithmen ein um eine konkrete Aufgabenstellung zu lösen
	\end{itemize}
  \item Benutzer
  	\begin{itemize}
  		\item Eigentümer
  		\item Anwender
  		\item Person deren Daten verwendete wird
  		\item Anspruchsgruppe (Stakeholder)
	\end{itemize}
\end{itemize}

\chapter{Fallbeispiel Katzenklappe}

\chapter{Bilderkennung}
Neuronale Netze, insbesondere \Gls{DNN}
\section{Heatmaps}
\imgText{dog-good-heatmap.png}{ 
Ein präzises Model verwendet keine Bildpunkte welche nicht dem gesuchten Objekt zugehören.
}

\imgText{dog-bad-heatmap.png}{ 
Obwohl in diesem Beispiel das Objekt korrekt erkannt wurde ist die präzision des Models klein.
}

\imgText{cat-good-heatmap.png}{ 
Allerdings ist auch eine gute Übereinstimmung mit dem Objekt keine Garantie dass die richtige Klasse gefunden wird.
}

\imgText{cat-bad-heatmap.png}{ 

}

\section{LRP}
\Gls{LRP} ist eine Technik 
% List of Figures
\listoffigures
% List of Tables
\begingroup
\let\clearpage\relax
\listoftables
% Glossary
\printglossary
% Bibliography

\renewcommand\bibname{Literaturverzeichnis}
%Index
\addcontentsline{toc}{chapter}{Index}
%\printindex
% Appendices

\bibliography{references} 
\bibliographystyle{apalike}

\endgroup
\end{document}
